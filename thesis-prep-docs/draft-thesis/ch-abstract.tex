\cleardoublepage
\pagenumbering{roman}
\setcounter{page}{5}
\begin{center}
{\LARGE \textsc{Abstract}}
\end{center}
\vspace{1cm}

Colorectal cancer remains one of the leading causes of cancer-related mortality worldwide, with early detection of polyps during colonoscopy being critical for prevention. However, manual polyp detection faces challenges including high miss rates (up to 26\%), inter-observer variability, and physician fatigue during lengthy procedures. This thesis presents a deep learning-based automated polyp detection system using the state-of-the-art YOLOv8 (You Only Look Once version 8) object detection framework.

The proposed system implements a complete end-to-end pipeline for polyp detection in colonoscopy videos, encompassing data preprocessing, model training, real-time inference, and comprehensive evaluation. The Kvasir-SEG dataset, containing 1,000 colonoscopy images with corresponding segmentation masks, was converted to YOLO bounding box format using custom preprocessing scripts that support both single and multi-component polyp detection through connected component analysis.

The YOLOv8-nano architecture was trained for 50 epochs on an 80:20 train-validation split, achieving outstanding performance metrics: 89.4\% mAP@50, 70.7\% mAP@50-95, 82.8\% precision, and 86.4\% recall. These results significantly exceed the target threshold of 70\% mAP@50, demonstrating the model's effectiveness for medical polyp detection.

Extensive validation was performed on real medical endoscopy videos across multiple polyp morphologies including MSD variants, pedunculated polyps, and ileocecal valve lesions. The system demonstrated consistent high-confidence detection (93-95\% maximum confidence) across all polyp types, with real-time processing capabilities of 30-60 FPS, making it suitable for clinical deployment.

The implementation provides a production-ready framework with comprehensive documentation, including scripts for data conversion, training, video inference with CSV logging, and automated evaluation. The system generates both annotated videos with bounding box visualizations and structured detection logs for medical analysis and record-keeping.

This work demonstrates the viability of YOLO-based object detection for automated polyp screening, offering potential to reduce miss rates, assist gastroenterologists during procedures, and serve as an educational tool for medical training. The complete codebase, trained model weights, and validation results are provided as a reproducible research package.

\vspace{0.5cm}
\noindent \textbf{Keywords:}~Deep Learning, Polyp Detection, YOLOv8, Object Detection, Medical Image Analysis, Colonoscopy, Computer Vision, Convolutional Neural Networks, Real-time Detection, Colorectal Cancer Screening
